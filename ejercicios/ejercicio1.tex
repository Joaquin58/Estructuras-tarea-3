\chapter*{1) Asumiendo los axiomas de un álgebra booleana  A = \{{0,1},+,$\cdot$\} demostrar las siguientes propiedades:}
\begin{enumerate}[label=\alph*)]
	\item Idempotencia: $x+x = x $ y $ xx = x$.
	\item Idempotencia de complemento:  $(\bar{\bar{x}})= x$.
	\item Elemento dominante: $x+1 = 1$ y $x0 = 0$.
	\item Absorción: $x+xy = x$ y $x(x+y) = x$.
\end{enumerate}
\begin{multicols}{2}
	\noindent
	$\underline{Dem \;a)}:\; \text{Sea $x\in A$ un elemento del álgebra booleana}$
	\begin{align*}
		(x+x) & =(x+x)\cdot 1           \\
		      & =(x+x)\cdot (x+\bar{x}) \\
		      & =x + x\bar{x}           \\
		      & =x + 0                  \\
		      & =x\; \blacksquare       \\
	\end{align*}

	\columnbreak

	\noindent
	$\underline{Dem}:\; \text{Sea $x\in A$ un elemento del álgebra booleana}$
	\begin{align*}
		xx & =xx+0              \\
		   & =xx+(x\bar{x})     \\
		   & =x\cdot(x+\bar{x}) \\
		   & =x\cdot 1          \\
		   & =x                 \\
	\end{align*}
\end{multicols}
\begin{multicols}{2}
	\noindent
	$\underline{Dem \;b)}:\; \text{Sea $x\in A$ un elemento del álgebra booleana}$
	\begin{align*}
		\text{Supongamos } \bar{\bar{x}}\neq x                                                                               \\
		                                                            & x+\bar{x}=1; x\cdot{\bar{x}}=0                         \\
		                                                            & \bar{x}+\bar{\bar{x}}=1; \bar{x}\cdot{\bar{\bar{x}}}=0 \\
		x+\bar{x}=                                                  & \bar{x}+\bar{\bar{x}}=1                                \\
		\text{Unicidad del complemento, }x \text{ y } \bar{\bar{x}} & \text{ son complemento de } \bar{x}                    \\
		\implies x = \bar{\bar{x}} \mathrel{\text{\large!}}                                                                  \\
		\therefore \bar{\bar{x}}= x\;\blacksquare
	\end{align*}
\end{multicols}

\begin{multicols}{2}
	\noindent
	$\underline{Dem \;c)}:\; \text{Sea $x\in A$ un elemento del álgebra booleana}$
	\begin{align*}
		(x+1) & = x+(x+\bar{x}) \\
		      & = (x+x)+\bar{x} \\
		      & = x+\bar{x}     \\
		      & = 1
	\end{align*}

	\columnbreak

	\noindent
	$\underline{Dem}:\; \text{Sea $x\in A$ un elemento del álgebra booleana}$
	\begin{align*}
		x\cdot0 & = x\cdot(x\cdot \bar{x})  \\
		        & = (x\cdot x)\cdot \bar{x} \\
		        & = (x\cdot x)\cdot \bar{x} \\
		        & = x\cdot \bar{x}          \\
		        & = 0
	\end{align*}
\end{multicols}

\begin{multicols}{2}
	\noindent
	$\underline{Dem \;d)}:\; \text{Sea $x\in A$ un elemento del álgebra booleana}$
	\begin{align*}
		x+xy            & =x \\
		(x+x)(x+y)      & =  \\
		x(x+y)          & =  \\
		x(1\cdot (1+y)) & =  \\
		x(1\cdot 1)     & =  \\
		x(1)            & =  \\
		x               & =x \\
	\end{align*}

	\columnbreak
	\noindent
	$\underline{Dem}:\; \text{Sea $x\in A$ un elemento del álgebra booleana}$
	\begin{align*}
		x(x+y)             & =x  \\
		x\cdot x +x\cdot y & =   \\
		x +x\cdot y        & =   \\
		x (1 + 1\cdot y)   & =   \\
		x (1)              & =   \\
		x                  & = x
	\end{align*}
\end{multicols}
